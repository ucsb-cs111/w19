\documentclass[11pt]{article}

\newcommand{\numpy}{{\tt numpy}}    % tt font for numpy

\topmargin -.75in
\textheight 8.5in
\oddsidemargin  -.25in
\evensidemargin -.20in
\textwidth 7in

\begin{document}

$$\mbox{\Large \bf CS 111: Homework 3: Due by 6:00pm Friday, January 25}$$
\par\bigskip\noindent
{\bf NEW TURNIN RULES:} Homework must be submitted online as a PDF file to GradeScope.
When you turn in your homework, tell GradeScope which page(s) of your PDF contain
each individual problem. Doing this correctly will be worth 2 points.

\par\bigskip
{\bf 1.}
How many arithmetic operations 
(total of additions, subtractions, multiplications, divisions)
are required to do each of the following?
(You can omit lower-order terms in $n$.)

\par\medskip
{\bf 1a.} Compute the sum of two $n$-vectors?

\par\medskip
{\bf 1b.} Compute the product of an $n$-by-$n$ matrix with an $n$-vector?

\par\medskip
{\bf 1c.} Compute the product of two $n$-by-$n$ matrices?

\par\medskip
{\bf 1d.} Solve an $n$-by-$n$ upper triangular linear system $Ux=y$?


\par\bigskip
{\bf 2.}
Suppose $A$ and $B$ are $n$-by-$n$ matrices, with $A$ nonsingular,
and $c$ is an $n$-vector.
Describe the steps you would use to {\em efficiently} compute the
product $A^{-1}Bc$.
Describe a different, less efficient, sequence of steps.


\par\bigskip
{\bf 3.}
Suppose that $A$ is a square, nonsingular, nonsymmetric matrix, 
$b$ is an $n$-vector, and that you have called 
$$\mbox{\tt L, U, p = LUfactor(A)}$$
(using the routine from the lecture files).
Now suppose you want to solve the system $A^Tx=b$ (not $Ax=b$) for $x$.
Show how to do this using calls to {\tt Lsolve()} and {\tt Usolve()},
without modifying either of those routines or calling {\tt LUfactor()} again.
Test your method in {\tt numpy} on a randomly generated 6-by-6 matrix
(see {\tt np.random.rand()}).


\par\bigskip
{\bf 4.} Do problem 2.3 on pages 32--33 of the NCM book, 
showing the {\tt numpy} code you use and its output. 
Note: To understand intuitively what the problem means by 
``assume that joint 1 is rigidly fixed both horizontally and vertically 
and that joint 8 is fixed vertically,'' 
think of the truss as a (2-dimensional) drawbridge across a river, 
with the left end being a hinge and the right end lying on the ground.


\par\bigskip
{\bf 5.} Consider the linear system
$$
   \left(
   \begin{array}{cc}
      \alpha & 1 \\ 	
           1 & 1 \\ 	
   \end{array} \right)
   \left(
   \begin{array}{c}
      x_0 \\ 	
      x_1 \\ 	
   \end{array} \right)
   =
   \left(
   \begin{array}{cc}
      \alpha + 2 \\ 	
               3 \\ 	
   \end{array} \right),
$$
for some $\alpha < 1$.
Clearly the solution is $(x_0, x_1)^T = (1,2)^T$.
For each value of $\alpha = 10^{-4}, 10^{-8}, 10^{-16}, 10^{-20}$,
solve this system using the routines {\tt LUfactor()}, {\tt Lsolve()}, 
and {\tt Usolve()} from {\tt LUsolve.ipynb} in the lecture files.
For each $\alpha$, do this twice, first with {\tt pivoting = True} in {\tt LUfactor()}
and then with {\tt pivoting = False}.
Show your {\tt numpy} code and its output.
Comment on your results.


\par\bigskip
{\bf 6.} Recall that a symmetric matrix $A$ is {\em positive definite}
(SPD for short) if and only if $x^TAx>0$ for every nonzero vector $x$.

\par\medskip
{\bf 6a.} Find a 2-by-2 matrix $A$ that (1) is symmetric, (2) is not singular,
and (3) has all its elements greater than zero, but (4) is {\em not} SPD.
Show a nonzero vector $x$ such that $x^TAx<0$.

\par\medskip
{\bf 6b.} Let $B$ be a nonsingular matrix, of any size, 
not necessarily symmetric.
Prove that the matrix $A=B^TB$ is SPD.

\end{document}
