\documentclass[11pt]{article} 

\usepackage{amssymb,amsmath}

\newcommand{\numpy}{{\tt numpy}}            % tt font for numpy
\newcommand{\scipy}{{\tt scipy}}            % tt font for scipy
\newcommand{\matplotlib}{{\tt matplotlib}}  % tt font for matplotlib

\topmargin -1in
\textheight 9in
\oddsidemargin  -.25in
\evensidemargin -.20in
\textwidth 7in

\begin{document}

$$\mbox{\Large \bf CS 111: Homework 8: Due by 6:00pm Friday, March 15}$$
\par\bigskip\noindent
Both problems on this week's homework are from Michael Heath's 
{\em Scientific Computing: An Introductory Survey}, 
which is a really superb textbook for a course at a slightly higher level than CS 111.

\par\smallskip
Homework must be submitted online as a PDF file to GradeScope.
When you turn in your homework, 
tell GradeScope which page(s) contain each problem. 
Doing this correctly will be worth 2 points.

\par\bigskip
{\bf 1.} 
A variation of the {\em Kermack-McKendrick model} 
for the course of an epidemic in a population is given by the system of ODEs
\begin{align}
\dot y_0 &= -cy_0y_1, \\
\dot y_1 &= cy_0y_1 - ry_1 - dy_1, \\
\dot y_2 &= ry_1, \\
\dot y_3 &= dy_1,
\end{align}
where $y_0$ represents susceptibles,
$y_1$ represents infectives in circulation,
$y_2$ represents infectives removed by recovery and immunity,
and $y_3$ represents infectives removed by death.
The parameters $c$, $r$, and $d$ represent the infection rate,
recovery rate, and death rate, respectively.

Use {\tt integrate.solve\_ivp()} to solve this system numerically,
with the parameter values $c=1$, $r=5$, and $d=1$,
and initial values $y_0(0)=95$, $y_1(0)=5$, and $y_2(0)=y_3(0)=0$.
Solve for times $t=0$ to $t=1$.
Plot all four components of $y$ on the same graph as a function of $t$,
using {\tt plt.legend()} to show which curve is which,
and labeling the plot axes appropriately.
As expected with an epidemic, 
you should see the number of infections grow at first,
then diminish to zero.

Experiment with other values for the parameters and initial conditions.
Can you find values for which the epidemic never infects the whole population,
or for which the entire population is wiped out?
Turn in plots of your results along with an explanation in English
of what you tried and what you discovered.

\par\bigskip
{\bf 2.} 
An important problem in classical mechanics is to determine the motion
of two bodies under mutual gravitational attraction.
Suppose that a body of mass $m$ is orbiting a second body of much larger mass $M$,
such as the earth orbiting the sun.
From Newton's laws of motion and gravitation,
the orbital trajectory $(x_0(t), x_1(t))$ is described by the system
of second-order ODEs
\begin{align}
\ddot x_0 &= -GMx_0/r^3, \\
\ddot x_1 &= -GMx_1/r^3, 
\end{align}
where $G$ is the gravitational constant and $r = (x_0^2 + x_1^2)^{1/2} = ||x||$
is the distance of the orbiting body from the center of mass of the two bodies.
For this exercise, we choose units such that $GM=1$.

\par\medskip
{\bf 2a.}
Use {\tt integrate.solve\_ivp()} to solve this system of ODEs with the initial conditions
\begin{align}
x_0(0)      = 1-e, & \;\;\; x_1(0)      = 0, \\
\dot x_0(0) = 0,   & \;\;\; \dot x_1(0) = \Big(\frac{1+e}{1-e}\Big)^{1/2},
\end{align}
where $e$ is the eccentricity of the resulting elliptical orbit,
which has period $2\pi$.
Try the values $e=0$ (which should give a circular orbit),
$e=0.5$, and $e=0.9$.
For each case, solve the ODE for at least one orbital period and obtain output 
at enough intermediate points to draw a smooth plot of the orbital trajectory.
Make separate plots of $x_0$ versus $t$, $x_1$ versus $t$, and $x_0$ versus $x_1$,
all with well-labeled axes and clear titles.
For your plot of the orbit itself, $x_0$ versus  $x_1$, 
use {\tt plt.gca().axis('equal')} to make sure the scale is the same on both axes,
so that a circle will look like a circle.

Experiment with different error tolerances 
(use {\tt help(integrate.solve\_ivp)} to find out how to set error tolerances)
to see how they affect 
(i)  the amount of time required for the solution and 
(ii) how close the orbit comes to being closed.
If you trace the orbit through several periods, 
does the orbit tend to wander or remain steady?
In addition to your plots, 
turn in an explanation in English of what experiments you did,
what you observed, and what your conclusions were.

\par\medskip
{\bf 2b.}
Physics tells us that the orbit described by this ODE
(like any closed physical system) should conserve both energy and angular momentum.
Check your numerical solutions from part (a) to see how well the following
quantities remain constant.

Conservation of energy:
$$\frac{\dot x_0^2 + \dot x_1^2}{2} - \frac{1}{r}.$$

Conservation of angular momentum:
$$x_0\dot x_1 - x_1\dot x_0.$$

Present your results using whatever numbers or graphs you think
are most informative, and describe your conclusions in English.

\end{document}
